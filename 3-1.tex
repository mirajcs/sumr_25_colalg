\documentclass{beamer}

\usepackage{amsmath, amssymb}
\usepackage{tikz-cd}
\usepackage{xcolor}
\usepackage{graphicx}

\title{MAT102 - College Algebra - Polynomial and Rational Functions}
\subtitle{3.1 Quadratic Functions and Applications \cite{miller2016college}}
\author{\textbf{Miraj Samarakkody}}
\institute{Tougaloo College}
\date{Updated - \today}

\begin{document}

\begin{frame}
    \titlepage
\end{frame}

\begin{frame}{Graph a Quadratic Function Written in Vertex Form}
    \begin{itemize}
        \item A function of the form \(f(x) =mx+c ~ (m \ne 0)\) is a linear function. \pause
        \item The function defined by \(f(x) = ax^2 +bx + c ~(a \ne 0) \) is called a \textbf{quadratic function.}  \pause
    \end{itemize}
    \includegraphics[scale=0.5]{figs/Figure_1.png}\pause
\end{frame}

\begin{frame}
    \frametitle{Graph a Quadratic Function Written is Vertex Form}
    \includegraphics[scale=0.5]{figs/Figure_2.png}

    

\end{frame}

\begin{frame}
    \frametitle{Graph a Quadratic Function Written is Vertex Form}
    \includegraphics[scale=0.5]{figs/Figure_3.png}
\end{frame}

\begin{frame}
    \frametitle{Quadratic Function}

    A function defined by \(f(x) = ax^2 +bx +c~ (a \ne 0)\) is called a \textbf{quadratic function}. By completing the square, \(f(x)\) can be expressed in \textbf{vertex form} as \(f(x)= a(x-h)^2+k\).\pause
    \begin{itemize}
        \item The graph of \(f\) is a parabola with vertex \((h,k)\). \pause
        \item If \(a>0\), the parabola opens upward, and the vertex is the minimum point. The minimum value of \(f\) is \(k\). \pause
        \item If \(a<0\), the parabola opens downward, and the vertex is the minimum point. The minimum value of \(f\) is \(k\). \pause
        \item The axis of symmetry is \(x=h\). This is the vertical line that passes through the vertex. 
    \end{itemize}

\end{frame}

\begin{frame}{Example - Analyzing and Graphing a Quadratic Function}

    
    
\end{frame}





\begin{frame}
    \frametitle{References}
    \bibliographystyle{plain} % or another style like unsrt, alpha, etc.
    \bibliography{reference}  % omit the .bib extension
\end{frame}

\end{document}

